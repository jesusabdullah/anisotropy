\chapter{Conclusions}

An analytical model based on the isotropic solution outlined by
Carslaw and Jaeger has been modified in order to predict thermal conductivity
measurements of anisotropic materials as a function of insertion angle.  This
model uses a linear transformation to pose the problem in a form equivalent to
the isotropic problem. However, in order to get meaningful results, the model
requires calculating the average temperature along an ellipse.

A 3-D finite element numerical model has also been built in order to predict
thermal conductivity measurements  of anisotropic materials as a function of
insertion angle. While the base model is simple by finite element model
standards, the number of parameters iterated through is somewhat unusual for
finite element modeling, and taxes the abilities of the software used.

According to both numerical and analytical theories, anisotropic thermal
conductivity should cause predictable changes in needle probe heating curve
measurements as a function of angle. While the two models show similar trends, 
there are significant differences between the two predictions. These
may in part be explained by the 3D model's handling of edge effects, and in part
due to the numerical model having too coarse of a mesh to accurately model the
problem.

Measurements of engineered anisotropic materials are promising, but far from
complete. A basic, repeatable method has been designed and tested. The initial
results indicate the expected trend; however, due to the relatively low
amount of anisotropy in the material, the variance in measurements of the thermal
conductivity measurement method and a disappointingly low amount of
measurements, a null hypothesis is impossible to rule out.

Similarly, in-situ measurements of snow have been made, but due to difficulties
in snow measurement very few useful data points were collected. However, the
results do indicate anisotropy in the snow. This detected anisotropy is likely
due to aggregate effects of layering, and not due to the structural anisotropy
that could potentially explain the differences between guarded hot plate and
needle probe measurements.

Based on these results, anisotropy is likely detectable in snow and possible to
determine, though it will require more measurements than perhaps hoped for. If
this is anisotropy is to explain the discrepancies between guarded hot plate and
needle probe measurements, it must be shown that structural anisotropy
in snow is significant enough not only to explain the guarded hot plate/needle
probe discrepancies itself, but also enough to counteract any anisotropy being
caused by layering in snow. This research suggests that structural anisotropy
\emph{may} be able to explain discrepancies between guarded hot plate
measurements and needle probe measurements, though, with a lack of hard date,
the answer to this problem is still unknown. More experiments will be
required to know for sure.

Inquiry on this method is far from complete. First, there are significant
descrepancies between the analytical and numerical theories which must be resolved.
Second, there are not enough measurements to benchmark either method. Because
of these issues, a suitable, robust method for ascertaining anisotropic thermal
conductivity from multiple needle probe measurements has not been sufficiently
developed. Development of such a method will be instrumental not only in
benchmarking the theoretical predictions but also critical in answering the
question of what causes the disagreement between guarded hot plate and
needle probe measurements.
