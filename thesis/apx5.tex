\chapter{Needle Probe Apparatus Directions}

\section{Taking A Measurement}

The apparatus is controlled using a keypad that enables one to communicate
with the data logger using Campbell ``star-codes.'' Initiating a test is a
matter of clearing and setting some registers using these star-codes.

In general, the apparatus is used like so:

\begin{enumerate}
\item Turn on the device.
\item Insert the needle into medium being measured.
\item Use star-codes to clear the first three registers. ``* 6 A'' accesses the
registers, and ``D \(n\)'' toggles the \(n\)th register. For example, to clear
the second register, press ``* 6 A D 2'' .
\item Turn on the first register by pressing ``* 6 A D 1''. This makes the
apparatus measure temperature.
\item Turn on the second register by pressing ``* 6 A D 2''. This turns on the
heating element, effectively starting the test.
\item Wait 20 minutes for test to complete.
\end{enumerate}

\section{CSV Headers}

Data from the Campbell instrument comes in the form of a .csv file. For this particular experiment, the columns (from left
to right) represent:

\begin{enumerate}
\item An instrument ID (constant in this case).
\item Ordinal day, out of 366. For example, March 17th is day 76.
\item hh:mm portion of timestamp. For example, 6:30pm is represented as 1830.
\item Seconds portion of timestamp.
\item Needle temperature, in Celcius.
\item Reference temperature, in Celcius.
\item Voltage across needle probe, in millivolts.
\item Experiment timer, in seconds.
\end{enumerate}
