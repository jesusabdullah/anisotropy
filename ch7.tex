\chapter{Conclusions}

An analytical model based on the isotropic solution outlined by
Carslaw and Jaeger has been modified in order to predict thermal conductivity
measurements of anisotropic materials as a function of insertion angle.  This
model uses a linear transformation to pose the problem in a form equivalent to
the isotropic problem. However, in order to get meaningful results, the model
requires calculating the average temperature along an ellipse.

A 3-D finite element numerical model has also been built in order to predict
thermal conductivity measurements  of anisotropic materials as a function of
insertion angle. While the base model is simple by finite element model
standards, the number of parameters iterated through is somewhat unusual for
finite element modeling, and taxes the abilities of the software used.

According to both numerical and analytical theories, anisotropic thermal
conductivity should cause predictable changes in needle probe heating curve
measurements as a function of angle. While the two models show similar trends, 
there are significant differences between the two predictions. These
may be explained by the 3D model's handling of edge effects, as well as possible
issues with the numerical model's convergence properties.

Measurements of engineered anisotropic materials are promising, but far from
complete. A basic, repeatable method has been designed and tested. The initial
results indicate the expected trend; however, due to the relatively low
amount of anisotropy in the material, the variance in measurements of the thermal
conductivity measurement method and a disappointingly low amount of
measurements, a null hypothesis is impossible to rule out.

Similarly, in-situ measurements of snow have been made, but due to difficulties
in snow measurement very few useful data points were collected. However, the
results do suggest anisotropy in the snow.

Inquiry on this method is far from complete. Aside from the need
for more measurements, there are plenty of avenues for future researchers to
study. These include updating and refining the methods, studying cooling curves,
and analyzing the current models with increased scrutiny.
