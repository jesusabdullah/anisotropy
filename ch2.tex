\chapter{Analytical Needle Probe Approach}
\label{sec:analytical-np}
\bigskip

\section{Introduction}
\label{sec:analytical-np:intro}

The technique used to measure thermal conductivity with a needle probe is based on the assumption that the needle approximates an infinite line source of energy with a constant heat flux embedded in an infinite medium. The origins of the analytical solution for isotropic thermal conductivity may be found in Carslaw and Jaeger's book, ``Conduction of Heat in Solids.'' \marginpar{Need to bibtex up my sources!}

The method based on Carslaw and Jaeger's solution depends on solving a 2-D problem, where all planes orthogonal to the needle have the same temperature distribution; in other words, temperature is not a function of axial position. Moreover, the problem is further simplified by posing the problem into radial coordinates and solving for temperature as a function of radial distance only.

In the isotropic case, this is straightforward, as conductivity is a constant scalar. Unfortunately, the anisotropic case is more complex, but luckily not completely intractible.

\section{The Isotropic Case}
\label{sec:analytical-np:isotropic}

The isotropic case solves the following equation:

\begin{equation*}
-k\nabla^2 T = \rho C\frac{\partial T}{\partial t}
\end{equation*}

Where \(T\) is temperature, \(k\) is a scalar thermal conductivity, \(\rho\) is density, \(C\) is volumetric heat capacity, and \(t\) is time.

\marginpar{From Carslaw \& Jaeger, pg. 261}

By casting this problem into cylindrical coordinates, the equations may be simplified such that they are a function of radial distance \(r\) only (as temperature is assumed to not be a function of axial position \(z\) or angle \(\phi\)).

After applying this transformation and solving the equation, the analytical solution to the problem becomes:

\begin{equation}
\label{isotropic_ei}
T(r,t) = - \frac{q}{4\pi k}\Ei\left(-\frac{r^2}{4kt}\right)
\end{equation}

where \(q\) is heat flux from the needle per linear distance.

Solving for the exponential integral analytically is not possible, and numerical solutions can be difficult. Typically, we instead use an approximation for small \(r/t\),

\begin{equation}
\label{isotropic_case}
T(r,t) = \frac{q}{4\pi K}\ln\left(\frac{4Kt}{r^2}\right) - \frac{\gamma q}{4\pi K}
\end{equation}

Typical use of this function is to find \(\frac{dT}{d(\ln t)}\) and solve that for \(k\). We will be working with \ref{isotropic_case} for the remainder of this analysis, though it could easily be applied to \ref{isotropic_ei} as well.


\section{Difficulties In The Anisotropic Case}
\label{sec:analytical-np:anisotropic-diff}


The anisotropic case varies from the isotropic one in that instead of a scalar thermal conductivity \(k\), one must solve the problem using an \(n \times n\) thermal conductivity, \([K]\), where \(n\) is the number of dimensions in the problem. As a consequence, reducing the problem into two dimensions becomes more difficult. Moreover, when the problem is posed in cylindrical coordinates, the solution becomes a function not only of \(r\), but of \(\phi\) as well.

\section{Posing The Problem in Two Coordinates}
\label{sec:analytical-np:2D}

By assuming that temperature distribution is not a function of axial direction \(z\), we may reduce the problem to an analogous one in orthogonal directions \(x\) and \(y\) instead:

\begin{equation}
-\nabla_{xy} \cdot \left([K]_{2 \times 2}\nabla_{xy} T \right)= \rho C\frac{\partial T}{\partial t}
\end{equation}

Without loss of generality, it may be further simplified like so:

\begin{equation}
-\nabla \cdot \left(\begin{bmatrix}k_x & 0\\ 0 && k_y\end{bmatrix}\nabla T \right)= \rho C\frac{\partial T}{\partial t}
\end{equation}

We are able to do this by choosing the directions \(x\) and \(y\) such that the matrix is diagonal.

The values of \(k_x\) and \(k_y\) may be found by finding the components of \([K]\) that are in the \(xy\) plane, and then finding the eigenvalues of the resulting components. \marginpar{This needs to be formalized. I have python code somewhere, which may also be a good thing to take snippets of.}


\section{Coordinate Transformation}
\label{sec:analytical-np:transformation}

In order to apply the isotropic solution to this anisotropic case, we would like to apply a coordinate transformation such that we can transform the problem into an isotropic case with respect to some \(x' = a_x x\) and \(y' = a_y y\). Without loss of generality, suppose \(a_x = 1\).

\begin{align}
\frac{dx'}{dx} &= 1\\
\frac{dy'}{dy} &= a_y\\
\frac{\partial f}{\partial x} &= \frac{\partial f}{\partial x'}\frac{dx'}{dx} = \frac{\partial f}{\partial x'}\\
\frac{\partial f}{\partial y} &= \frac{\partial f}{\partial y'}\frac{dy'}{dy} = a_y\frac{\partial f}{\partial y'}\\
\nabla T &= \frac{\partial T}{\partial x'} \e_{x'} + a_y\frac{\partial T}{\partial y'} \e_{y'} \\
[K]\nabla T &= k_x\frac{\partial T}{\partial x'} \e_{x'} + k_ya_y\frac{\partial T}{\partial y'} \e_{y'}\\
\nabla \cdot \left([K]\nabla T\right) &= k_x\frac{\partial^2 T}{\partial {x'}^2} + k_ya_y^2\frac{\partial^2 T}{\partial {y'}^2}\\
\end{align}

Suppose we set the right side equal to the equivalent isotropic expression:
\begin{equation*}
k\left(\frac{\partial^2 T}{\partial {x'}^2} + \frac{\partial^2 T}{\partial {y'}^2} \right) = k_x\frac{\partial^2 T}{\partial {x'}^2} + k_ya_y^2\frac{\partial^2 T}{\partial {y'}^2}
\end{equation*}

As a result,

\begin{align*}
k &= k_x\\ a_y &= \sqrt{\frac{k_x}{k_y}}\\
\end{align*}

Therefore, the following coordinate transformation will allow us to apply the
isotropic solutions to an isotropic case with \(k = k_x\):

\begin{equation}
    \label{coord_trans}
    \begin{pmatrix}x' \\ y'\end{pmatrix} =
    \begin{bmatrix}1 & 0\\ 0 & \sqrt{\frac{k_x}{k_y}} \end{bmatrix}\begin{pmatrix}x \\ y\end{pmatrix}
\end{equation}

\section{From Temperature Distribution to Measurement}
\label{sec:analytical-np:isotropic}

Using equation \ref{coord_trans}, we may apply the isotropic solution:

\begin{equation}
    -k_x \nabla^2 T = \rho C\frac{\partial T}{\partial t}
\end{equation}

and get the following result (for sufficiently large \(t/r'\)):

\begin{equation}
T(r',t) = \frac{q}{4\pi k_x}\ln\left(\frac{4k_xt}{r'^2}\right) - \frac{\gamma q}{4\pi k_x}
\end{equation}

When measuring for the anisotropic case, I argue that we are effectively measuring the temperature at some \(r = r_{\textrm{0}}\)---either on the surface of the needle, or some small distance away from the needle.

Applying this technique to the anisotropic case, we find that we must also transform \(r_{xy} = \cos(\theta) \hat{e}_x + \sin(\theta) \hat{e}_y \) into \(r_{x'y'}\):

\begin{align*}
    \begin{pmatrix}r_{x'} \\ r_{y'}\end{pmatrix} &=
    \begin{bmatrix}1 & 0\\ 0 & \sqrt{\frac{k_x}{k_y}} \end{bmatrix}\begin{pmatrix}r_0\cos(\theta) \\ r_0\sin(\theta)\end{pmatrix}\\
    &= r_0\left(\cos(\theta) \e_x + \sqrt{\frac1{k_y}} \e_y \right)\\
\end{align*}
\begin{equation}
    \norm{r}^2 = r_0^2 \left(\cos^2(\theta) + \frac{k_x}{k_y}\sin^2(\theta) \right)\\
\end{equation}

This means that the temperature around the needle should now vary as a function of \(\theta\), unlike in the isotropic case. Now, since the needle only measures a single value, it seems logical to suggest that the measured quantity is an average temperature---say, the average surface temperature of the probe.  This may be expressed like so:

\begin{equation}
T_{\textrm{avg}}(t) = \frac{ \int_0^{2\pi} T(r,t) rd\theta }
                           { \int_0^{2\pi} rd\theta}
\end{equation}

I ended up simplifying it such that, where

\begin{equation}
\mathcal{E}(f(\phi, \alpha), \alpha) = \int_0^{2\pi} f\sqrt{\cos^2(\phi) + \alpha\sin^2(\phi)} d\phi
\end{equation}

we have

\begin{equation}
\frac{4\pi k_x}{q} \frac{\mathcal{E}(\ln(t), \frac{k_y}{k_x})}{\mathcal{E}(1, \frac{k_y}{k_x})}
\end{equation}

\section{Finding Measured Conductivity as a Function of Needle Orientation}
