\chapter{What Didn't Work}

\section{Introduction}

\section{Parallelism}

Last summer, after making the first iteration of my model, I found that
simulations took a while to run. At about the same time, I attended the
Scipy 2010 conference, and I saw a talk---I believe by David
Beasley---about ``DIY concurrency.'' Despite Beasley's obvious bias
towards python-based solutions, it was a really good talk.

I felt inspired.

I attempted to develop a scheme that would use SSH to implement an
ad-hoc mapreduce-esque scheme across a number of ARSC workstations to
split the job of calulating K-values amongst about six machines.

\begin{figure}[htbp]
\centering
\includegraphics{fig/parallelism.jpg}
\caption{}
\end{figure}

Of course, looking at this picture, you might realize how crazy this
idea was. Parallelism isn't intractible, but trying to implement
completely parallelism with a buggy COMSOL/MATLAB interface that doesn't
run headless very well and is hard to even start from the command line,
is a clunky, time-consuming preposition. In three months of hard work,
it still wasn't working.

Of course, due to licensing constraints on the part of COMSOL were only
one instance may be open at a time per person, it became a moot point.

The lesson here isn't that ad-hoc concurrency is a bad idea. The lesson
here is that one should be weary of complexity and that the importance
of strong foundations shouldn't be underestimated. People create
SSH-based concurrency schemes all the time, but they're also working
with software that works just fine and as-expected, without too many
kludges, on its own. Moreover, most concurrency uses pre-existing,
stable frameworks. If one wanted to run MATLAB programs concurrently,
for example, using The Mathworks' parallel MATLAB framework would likely
be the simplest, most robust way to do it.

If anything, what this hilights is the importance of time management
with research. Given the choice between just about \emph{anything} and
avoiding spending more time, the correct answer is to save time:

\begin{itemize}
\item
  \textbf{Spend time to make a process faster} or \textbf{Just do the
  process as-is}? The latter.
\item
  \textbf{Spend time doing something the Right Way} or \textbf{Just
  kludge it}? The kludge wins.
\item
  \textbf{Spend time shopping for the best buy} or \textbf{Buy the first
  thing you see that'll work?} Definitely the latter. That's what grant
  money's for!
\end{itemize}
It's a common pattern, and while it may seem to be less than ideal, it's
also realistic.

\section{Custom Data Logger}

Speaking of time management: I also spent a good chunk of time
researching the programming of my own data logger using a
microcontroller. This was because I wanted to measure angle with an
accelerometer and a digital compass, and working with the Campbell tools
is a pain \emph{even} given that all your instruments are analog in
nature.

This isn't in and of itself a bad idea. In fact, there would be many
advantages to using a custom-programmed microcontroller over a Campbell
data logger. However, there are other considerations:

\begin{enumerate}[1.]
\item
  I didn't know a lick about programming microcontrollers. I know a
  little more now, but not enough.
\item
  Someone had already set up the needle probe to work with a Campbell
  data logger.
\item
  There are less technically advanced ways to measure angle, such as
  protractors.
\end{enumerate}
The lesson here is to stay focused on one's project, and be
ever-consious of what's doable given a certain time period. In this
case, making a new apparatus would be a good project for an EE's senior
design project. However, for a Mechanical Engineering graduate student
that has a number of other goals, none of which actually call for making
a new measurement scheme for the needle probe, it's not as good of an
idea.

On the other hand, having followed the ``spend money'' rule, I now have
a MSP--430 that I don't need right now. This could lead to fun hackery
down the road. Who knows?

\section{Layered Glycerine}

One of the ``standards'' used with the needle probe apparatus is to
measure the thermal conductivity of glycerine in an insulated box.
Glycerine is chosen because its conductivity is similar to water without
convecting and because it automatically fills air gaps around the
needle. So, I considered using glycerine as a medium to build an
anisotropic material, figuring its gel-like nature would stop it from
mixing.

It's true that the gel stops mixing. However, what it doesn't stop is
buoyancy effects.

Consider, for a moment, how one would change the thermal conductivity of
a fluid. Basically the only thing you can do is add something else. So,
for example, I started by trying to add rubbing alcohol because rubbing
alcohol has about a third of the conductivity of water. On the other
hand, it's also about 70\% the density of water, and surely even a
smaller percentage as dense as glycerine. So, imagine trying to pour
heavy vegetable glycerine over a thinned, lightweight mixture of
glycerine and rubbing alcohol. What happens, of course, is that the
alcohol mixture floats on top of the unadulterated glycerine.

This shouldn't have been a surprise, and yet it was. Interesting how
that works. Current plans call for using salt and sugar (in powder form,
not in any solutions).
