\chapter{What Didn't Work}

This piece was written as a blog post for my blog
(http://jesusabdullah.github.com), but I felt that the information would be
useful for anybody wanting to continue this research.

\section{What Didn't Work}

Science tends to report in the affirmative. Most papers and
documentations focus on what \emph{did} work, and what \emph{was} shown
as a result. However, much of science is the ``error'' in ``trial and
error.'' Here, I try to document some of the ideas that didn't work, or
didn't end up being useful, and what could be learned from them.

\subsection{Ellipsoid Measurement Space}

When I first looked at this problem, I conjectured that the ``shape'' of
the measured conductivity space in spherical coordinates, where phi and
theta mirrored real-world orientation and radius corresponded to
measured conductivity, would have an ellipsoid shape. That is, I
suspected that the equation

\[
x^TKx = 1
\]
would hold, where x is \(||1/k||\) long
and in the measured direction, and \(K\) is the conductivity matrix.
However, this is \textbf{clearly} not the case.

This is, in part, not the case \emph{because} the needle isn't measuring
conductivity \emph{parallel} to the needle, but \emph{orthogonal} to the
needle, and some sort of average of said orthogonal conductivities at
that. In retrospect, this is obvious.

I suppose the lesson here is to make sure you fully understand the
problem before making wild conjecture.

That said, this \emph{is} a good method for conceptualizing eigenvalues,
given that your matrix is symmetric and positive definite. It's called
``Lame's Ellipsoid'' in certain contexts, and is used in discussing
equivalent energy states in certain dynamics problems and sometimes as
an alternative to Mohr's Circle. In fact, Mohr's Circle itself helps
visualize eigenvalues admirably, though in many problems outside
material mechanics a shear component doesn't make sense. I mean, ``shear
conductivity?'' What's that?

\subsection{Open-Source FEM Toolchain}

As I became somewhat disheartened over a lack of initial success with
COMSOL, I began exploring other tools. In particular, I was interested
in using open-source finite element tools. A number of tools \emph{do}
exist, but the situation isn't good, as I outlined in this reddit post,
likely written in December 2010:

\begin{quote}
I know this was posted months ago, but seeing that nobody really said
anything: I've done some research on open-source finite element analysis
and think I might be able to shed some light here.

Before I do, I noticed that you mentioned Abaqus, so I'll address
free-as-in-beer modeling first. All of the commercial codes cost money,
but their target audiences are also organizations and not people; As
such, with enough searching you may be able to find some cheap or free
student editions. One of my professors was shopping around an Abaqus
student edition that just limited you in terms of mesh size such that
you couldn't really do problems that required fine meshes in 3-D but
could get yourself going if a coarse solution was okay.

In terms of open-source software, here's my impression: It's out there,
but it's often unpolished and typically not all-inclusive the same way
commercial codes are. In a full FEA stack, you basically have four
parts:

\begin{itemize}
\item
  Modeling. This is when you say, ``my domains are shaped like this and
  act like that.''
\item
  Meshing. This is when you split your domains into little tetrahedrons
  (or whatever other shape makes sense).
\item
  FEA proper. This is when the program actually uses a finite element
  formulation of your governing equations to find a numerical solution.
  This is the part that requires iron.
\item
  Visualization/Post-Processing. Once a solution is obtained, it needs
  to be visualized somehow. This is where pretty pictures and graphs
  comes in.
\end{itemize}
In the open-source world, there are tons and tons of programs and codes
that can do all of these things, but they tend to be more decoupled and
have a higher learning curve. Some of them are better for some things
than for others. I haven't been successful in learning a full stack yet
and this list is far from exhaustive, but here are a few things I've ran
across:

\begin{itemize}
\item
  \href{http://www.salome-platform.org/}{Salome}, I believe, aims to be
  an interface to the full stack. That is, it uses third-party meshers
  and solvers to do the heavy lifting, and takes care of steps 1 and 4
  in a more common-style one-stop interface. Last time I tried to use it
  I found it pretty incomprehensible but it was also years ago so ymmv.
\item
  ELMER (http://elmerfem.org/) is a one-stop solution, I believe.
  It's modular in construction, but all the major pieces are there. On
  the other hand, while its learning curve isn't terrible and it has
  decent docs last I checked, I've also heard that its solvers aren't
  the best. Of course, for simple problems, simple solvers suffice.
\item
  GMSH (http://www.geuz.org/gmsh/) is a stand-alone mesher. Its
  style is a bit different from some 3-D modelers, but it seemed to work
  alright if you didn't need to force the mesh to be really tight in
  some areas and coarse in others.
\item
  MeshPy (http://mathema.tician.de/software/meshpy) looks good
  from meshing, and Andreas makes good software. However, there seems to
  be disappointingly little in the way of online documentation.
\item
  OpenFOAM (http://www.openfoam.com/) is, I believe, one of the
  standard solvers that Salome uses. It's a relatively common CFD
  package. I otherwise don't know much about it.
\item
  Dolfyn (http://www.dolfyn.net/index\_en.html) is another CFD
  package I heard about while researching a long time ago. I honestly
  don't know anything about it.
\item
  sfepy (https://github.com/rc/sfepy) should allow for FEA using
  python. However, the rest of the stack is up to you, and I don't think
  it's been proven ``in the wild.''
\item
  Paraview (http://www.paraview.org/) is a common visualization
  tool, and I believe many open source solvers output their results in
  the ``old-style'' paraview format.
\item
  Mayavi (http://code.enthought.com/projects/mayavi/ is another
  option for visualization.
\end{itemize}
If you've learned anything else about free modeling/analysis, I'm
definitely interested. Like I said, I'm pretty sure this is just
scratching the surface.

\end{quote}
At the time of authorship, these tools were much more fresh in my mind
than they are now, so I'll let Past Self speak in my stead.

\subsection{Parallelism}

Last summer, after making the first iteration of my model, I found that
simulations took a while to run. At about the same time, I attended the
Scipy 2010 conference, and I saw a talk---I believe by David
Beasley---about ``DIY concurrency.'' Despite Beasley's obvious bias
towards python-based solutions, it was a really good talk.

I felt inspired.

I attempted to develop a scheme that would use SSH to implement an
ad-hoc mapreduce-esque scheme across a number of ARSC workstations to
split the job of calulating K-values amongst about six machines.

\begin{figure}[H]
\centering
\includegraphics{fig/parallelism.jpg}
\caption{}
\end{figure}

Of course, looking at this picture, you might realize how crazy this
idea was. Parallelism isn't intractible, but trying to implement
completely parallelism with a buggy COMSOL/MATLAB interface that doesn't
run headless very well and is hard to even start from the command line,
is a clunky, time-consuming preposition. In three months of hard work,
it still wasn't working.

Of course, due to licensing constraints on the part of COMSOL were only
one instance may be open at a time per person, it became a moot point.

The lesson here isn't that ad-hoc concurrency is a bad idea. The lesson
here is that one should be weary of complexity and that the importance
of strong foundations shouldn't be underestimated. People create
SSH-based concurrency schemes all the time, but they're also working
with software that works just fine and as-expected, without too many
kludges, on its own. Moreover, most concurrency uses pre-existing,
stable frameworks. If one wanted to run MATLAB programs concurrently,
for example, using The Mathworks' parallel MATLAB framework would likely
be the simplest, most robust way to do it.

If anything, what this hilights is the importance of time management
with research. Given the choice between just about \emph{anything} and
avoiding spending more time, the correct answer is to save time:

\begin{itemize}
\item
  \textbf{Spend time to make a process faster} or \textbf{Just do the
  process as-is}? The latter.
\item
  \textbf{Spend time doing something the Right Way} or \textbf{Just
  kludge it}? The kludge wins.
\item
  \textbf{Spend time shopping for the best buy} or \textbf{Buy the first
  thing you see that'll work?} Definitely the latter. That's what grant
  money's for!
\end{itemize}
It's a common pattern, and while it may seem to be less than ideal, it's
also realistic.

\subsection{Direct use of MATLAB's \texttt{expint()} Function}

Interestingly enough, MATLAB actually comes with an exponential integral
function, though it's defined differently in MATLAB than it is in this
context. A failed approach involved trying to fit the data points
directly against the exponential integral solution. However, the
convergence properties of this approach were less than ideal.

\subsection{Custom Data Logger}

Speaking of time management: I also spent a good chunk of time
researching the programming of my own data logger using a
microcontroller. This was because I wanted to measure angle with an
accelerometer and a digital compass, and working with the Campbell tools
is a pain \emph{even} given that all your instruments are analog in
nature.

This isn't in and of itself a bad idea. In fact, there would be many
advantages to using a custom-programmed microcontroller over a Campbell
data logger. However, there are other considerations:

\begin{enumerate}[1.]
\item
  I didn't know a lick about programming microcontrollers. I know a
  little more now, but not enough.
\item
  Someone had already set up the needle probe to work with a Campbell
  data logger.
\item
  There are less technically advanced ways to measure angle, such as
  protractors.
\end{enumerate}
The lesson here is to stay focused on one's project, and be
ever-consious of what's doable given a certain time period. In this
case, making a new apparatus would be a good project for an EE's senior
design project. However, for a Mechanical Engineering graduate student
that has a number of other goals, none of which actually call for making
a new measurement scheme for the needle probe, it's not as good of an
idea.

On the other hand, having followed the ``spend money'' rule, I now have
a MSP--430 that I don't need right now. This could lead to fun hackery
down the road. Who knows?

\subsection{Layered Glycerine}

One of the ``standards'' used with the needle probe apparatus is to
measure the thermal conductivity of glycerine in an insulated box.
Glycerine is chosen because its conductivity is similar to water without
convecting and because it automatically fills air gaps around the
needle. So, I considered using glycerine as a medium to build an
anisotropic material, figuring its gel-like nature would stop it from
mixing.

It's true that the gel stops mixing. However, what it doesn't stop is
buoyancy effects.

Consider, for a moment, how one would change the thermal conductivity of
a fluid. Basically the only thing you can do is add something else. So,
for example, I started by trying to add rubbing alcohol because rubbing
alcohol has about a third of the conductivity of water. On the other
hand, it's also about 70\% the density of water, and surely even a
smaller percentage as dense as glycerine. So, imagine trying to pour
heavy vegetable glycerine over a thinned, lightweight mixture of
glycerine and rubbing alcohol. What happens, of course, is that the
alcohol mixture floats on top of the unadulterated glycerine.

This shouldn't have been a surprise, and yet it was. Interesting how
that works. Instead, the experiments use straight table salt and table
sugar.

\subsection{Conclusions}

Science is hard.
