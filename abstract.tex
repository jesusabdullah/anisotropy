REMEMBER TO WRITE THIS FROM SCRATCH!

Maximum length: 200 words.

\begin{comment}
A new method for measuring thermal conductivity is being adapted from the
standard method of measuring isotropic thermal conductivity with needle probes,
ASTM D5334, in order to enable the dertermination of an anisotropic thermal
conductivity. This method will have particular relevance to measuring thermal
conductivity of natural snowpacks where conductivity  can be strongly
anisotropic due to vertical thermal gradients, compaction and metamorphosis.
Needle probes, such as specified in ASTM D5334, consist of a thin needle
embedded with a constant-flux heating element and a thermocouple or thermistor.
Thermal conductivity of the surrounding material is calculated analytically
using needle temperature data as a function of time.

This new method uses finite element numerical solutions for apparent thermal
conductivity as a function of actual anisotropic thermal conductivity and needle
orientation. Values for anisotropic thermal conductivity are determined by
curve-fitting a function relating apparent results of the ASTM method at given
needle orientations and the actual solutions. The results of the numerical
simulations may then be applied to the measurement of actual anisotropic
materials. Unlike ASTM D5334, multiple measurements at different orientations
are required, and relies the number of free variables in the heat equation
(with an upper bound of six for a symmetric conductivity matrix, which may be
reduced with further simplifications).

The applicability and feasibility of this technique for improving arctic climate
models, which can strongly depend on snow conductivity, will be discussed.
\end{comment}
