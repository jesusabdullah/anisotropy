\chapter{Code Used for Chapter \ref{sec:numerical-np}}
\label{apx:numerical-np}

\section{worker.m}
\small
\begin{minted}{matlab}
%worker.m
%does the working

function worker(kxy,kz)
    load('angles.mat', 'angles');
    angles
    [kxy,kz] = meshgrid(kxy,kz);

    % The commented-out ``flreport'' gives one the option of suppressing
    % graphical output from COMSOL. This is useful if one wants to run
    % COMSOL in a headless manner. Unfortunately, COMSOL 3.5a on the ARSC
    % systems has extreme difficulties running in headless batch mode.
    %flreport('off');

    params=struct('rsnow', 0.4, ...
                  'rneedle', 0.00025, ...
                  'lneedle', 0.1, ...
                  'density_snow', 200, ...
                  'density_needle', 8000, ...
                  'cp_snow', 2050, ...
                  'cp_needle', 460, ...
                  'q_needle', 0.5, ...
                  'k_needle', 160, ...
                  'time', [logspace(0.1,1,15) logspace(1,3,15)], ...
                  'angles', angles );

    saveroot=['./solutions-' date '/'];

    mesh = mesher(0,params);
    for angle=angles,
        try
            solutions = arrayfun(@(x,y) solver(x,y,mesh,angle,params), ...
                            kxy,kz, 'UniformOutput', false);
            save solutions
            fprintf('Fitting solutions...\n');
            solutions = {cellfun(@(tsd) {fitter(tsd{1}(1,:),tsd{1}(2,:), ...
                                             0.999,params), ...
                             tsd{1}, tsd{2}}, ...
                         solutions, 'UniformOutput', false)};
            fprintf('A solution set just completed.');
            system([ 'echo "A solution set finished on" `hostname` ' ...
                     '| mutt -s "A solution set completed." ' ...
                     'josh.holbrook@gmail.com' ]);
        catch exception
            system([ 'echo "Exception occurred on" `hostname` ' ...
                     '| mutt -s "Exception occurred--' exception.message ...
                     '" josh.holbrook@gmail.com' ]);
        end
        angles = angles(2:length(angles));
        save('angles.mat', 'angles');
        %solutions
        mkdir(saveroot);
        save([ saveroot 'solution-' num2str(angle) ], ...
             'solutions','angle','kxy','kz','params');
    end

    % Emails me when everything's done
    system([ 'echo "Results completed on " `hostname` ' ...
             '| mutt -s "Results Completed" ' ...
             'josh.holbrook@gmail.com' ]);
    system('touch down');
end
\end{minted}
\normalsize


\section{mesher.m}
\small
\begin{minted}{matlab}
% COMSOL Multiphysics Model M-file
% Generated in part by COMSOL 3.5a 
% (COMSOL 3.5.0.608, $Date: 2009/05/11 07:38:49 $)
% the rest of it modified by Joshua Holbrook

function fem=mesher(angle,params)
    % mesh_generate(angle)
    % generates a mesh for the given angle. 

    fprintf(['meshing for angle=' num2str(angle) '...\n']);
    flclear fem

    % COMSOL version
    clear vrsn
    vrsn.name = 'COMSOL 3.5';
    vrsn.ext = 'a';
    vrsn.major = 0;
    vrsn.build = 608;
    vrsn.rcs = '$Name: v35ap $';
    vrsn.date = '$Date: 2009/05/11 07:38:49 $';
    fem.version = vrsn;

    % Geometry
    g1=sphere3(num2str(params.rsnow), ...
               'pos',{'0','0','0'}, ...
               'axis',{'0','0','1'}, ...
               'rot','0');
    g2=cylinder3(num2str(params.rneedle), ...
                 num2str(params.lneedle), ...
                 'pos',{num2str(-params.lneedle/2),'0','0'}, ...
                 'axis',{'1','0','0'}, ...
                 'rot','0');
    parr={point3(0,0,0)};
    g3=geomcoerce('point',parr);
    parr={point3(params.rsnow,0,0)};
    g4=geomcoerce('point',parr);
    parr={point3(0,params.rsnow,0)};
    g5=geomcoerce('point',parr);
    parr={point3(0,0,params.rsnow)};
    g6=geomcoerce('point',parr);
    parr={point3(-params.rsnow,0,0)};
    g7=geomcoerce('point',parr);
    parr={point3(0,-params.rsnow,0)};
    g8=geomcoerce('point',parr);
    parr={point3(0,0,-params.rsnow)};
    g9=geomcoerce('point',parr);

    % Analyzed Geometry (?)
    clear p s
    p.objs={g3,g4,g5,g6,g7,g8,g9};
    p.name={'ORIGIN','PT1','PT2','PT3','PT4','PT5','PT6'};
    p.tags={'g3','g4','g5','g6','g7','g8','g9'};

    s.objs={g1,g2};
    s.name={'SNOW','NEEDLE'};
    s.tags={'g1','g2'};

    fem.draw=struct('p',p,'s',s);
    fem.geom=geomcsg(fem);


    % ODE Settings
    clear ode
    clear units;
    units.basesystem = 'SI';
    ode.units = units;
    fem.ode=ode;


    % Initialize mesh
    fem.mesh=meshinit(fem, ...
                      'hauto',5, ...
                      'hgradsub',[2,1.1], ...
                      'hmaxsub',[2,0.0005]);

    % Refine mesh
    fem.mesh=meshrefine(fem, ...
                        'mcase',0, ...
                        'rmethod','longest');

    fem=multiphysics(fem);
end
\end{minted}
\normalsize

\section{solver.m}
\small
\begin{minted}{matlab}
% COMSOL Multiphysics Model M-file
% Generated by COMSOL 3.5a 
% (COMSOL 3.5.0.608, $Date: 2009/05/11 07:38:49 $)

function answer=solver(kxy,kz,fem,theta,params)
    % solver(kxy,kz,mesh,params)
    % uses comsol to pump out a solution using a given mesh-mat
    % and a k-matrix in comsol format.

    fprintf(['solving for kxy=' num2str(kxy) ...
             ' and kz=' num2str(kz) '...\n']);
    % Application mode 1
    clear appl
    appl.mode.class = 'GeneralHeat';
    appl.module = 'HT';
    appl.shape = {'shlag(1,''J'')','shlag(2,''T'')'};
    appl.sshape = 2;
    appl.assignsuffix = '_htgh';
    clear bnd
    bnd.type = {'q0','cont'};
    bnd.shape = 1;
    bnd.ind = [1,1,1,1,2,2,2,2,2,1,1,1,1,2];
    appl.bnd = bnd;
    clear equ
    equ.sdtype = 'gls';
    % densities
    equ.rho = {params.density_snow,params.density_needle};
    equ.init = 0;
    equ.shape = 2;
    % Heat capacities
    equ.C = {params.cp_snow,params.cp_needle};
    % Wattage
    equ.Q = {0,params.q_needle/pi/(params.rneedle)^2};
    % Heat conductivities
    arr = [cos(theta*pi/180), 0, sin(theta*pi/180);
           0, 1, 0;
           -sin(theta*pi/180), 0, cos(theta*pi/180)]; %rotation matrix
    equ.k = {symmetric_tocell( ...
                 arr*diag([kxy,kxy,kz])*(arr')), ...
                 params.k_needle};
    equ.ind = [1,2];
    appl.equ = equ;
    fem.appl{1} = appl;
    fem.frame = {'ref'};
    fem.border = 1;
    fem.outform = 'general';
    clear units;
    units.basesystem = 'SI';
    fem.units = units;

    % Coupling variable elements
    clear elemcpl
    % Integration coupling variables
    clear elem
    elem.elem = 'elcplscalar';
    elem.g = {'1'};
    src = cell(1,1);
    clear bnd
    bnd.expr = {{'T',{}},{'1',{}}};
    bnd.ipoints = {{'4',{}},{'4',{}}};
    bnd.frame = {{'ref',{}},{'ref',{}}};
    bnd.ind = {{'1','2','3','4','10','11','12','13'}, ...
               {'5','6','7','8','9','14'}};
    src{1} = {{},{},bnd,{}};
    elem.src = src;
    geomdim = cell(1,1);
    geomdim{1} = {};
    elem.geomdim = geomdim;
    elem.var = {'int_T','area'};
    elem.global = {'1','2'};
    elemcpl{1} = elem;
    fem.elemcpl = elemcpl;

    % ODE Settings
    clear ode
    clear units;
    units.basesystem = 'SI';
    ode.units = units;
    fem.ode=ode;

    % Multiphysics
    fem=multiphysics(fem);

    % Generate GMG mesh cases
    fem=meshcaseadd(fem,'mgauto','anyshape');

    % Extend mesh
    fem.xmesh=meshextend(fem);

    % Solve problem
    fem.sol=femtime(fem, ...
                    'solcomp',{'T'}, ...
                    'outcomp',{'T'}, ...
                    'blocksize','auto', ...
                    'tlist', params.time, ...
                    'estrat',1, ...
                    'tout','tlist', ...
                    'linsolver','gmres', ...
                    'itrestart',100, ...
                    'prefuntype','right', ...
                    'prefun','gmg', ...
                    'prepar',{'presmooth','ssor','presmoothpar',{'iter',3,'relax',0.8},'postsmooth','ssor','postsmoothpar',{'iter',3,'relax',0.8},'csolver','pardiso'}, ...
                    'stopcond','0.06-int_T/area', ...
                    'mcase',[0 1]);

    % Save current fem structure for restart purposes
    fem0=fem;

    % Plot solution
    %{
    postplot(fem, ...
             'slicedata',{'T','cont','internal','unit','K'}, ...
             'slicexspacing',5, ...
             'sliceyspacing',0, ...
             'slicezspacing',0, ...
             'slicemap','Rainbow', ...
             'solnum','end', ...
             'title','Time=100    Slice: Temperature [K]', ...
             'grid','on', ...
             'campos',[-2.636014311828346, ...
                       -3.4353207343472505, ...
                       2.4999999999999996], ...
             'camtarget',[0,0,0], ...
             'camup',[0,0,1], ...
             'camva',41.213465344831754);
    %}

    % Integrate
    T_thermistor=postint(fem,'T', ...
               'unit','K', ...
               'recover','off', ...
               'dl',8, ...
               'edim',0, ...
               'solnum','all');

    % Integrate
    T_surf_avg=postint(fem,'T', ...
               'unit','', ...
               'recover','off', ...
               'dl',[1,2,3,4,10,11,12,13], ...
               'edim',2, ...
               'solnum','end');

    answer={[fem.sol.tlist; T_thermistor],T_surf_avg};
    angles = params.angles(2:length(params.angles));

    %flsave(['fem-' num2str(theta) '-' num2str(kxy) ...
    %        '-' num2str(kz) '.mph']);

    save('angles.m', 'angles');
end
\end{minted}
\normalsize

\section{fitter.m}
\small
\begin{minted}{matlab}
function k = fitter(t,T,rset,params)
    logt = log(t(t>1));
    T = T(t>1);

    disp('Finding linear portion...');    
    for i=1:length(logt)-1
        C = corrcoef(logt(i:length(logt)), T(i:length(T)));
        r = sqrt(C(2,1));
        if r > rset %adjust this to get 'good' values
            disp(['linear fitting to ' num2str((length(logt)-i)) ...
                  ' points from t=' num2str(exp(logt(i))) ...
                  ' to t=' num2str(exp(logt(length(logt)))) '...']);
            x = polyfit(logt(i:length(logt)),T(i:length(T)), 1);
            break
        end
    end

    %plot(logt,T,'*');
    %hold on;
    %plot(logt, x(1)*logt + x(2));
    k = (params.q_needle)/(4*pi*x(1));

end
\end{minted}
\normalsize

\section{assembler.m}
\small
\begin{minted}{matlab}
function answers=assembler(directory)
    cd(directory);
    d = dir();
    answers = [];
    for i=3:length(d),
        disp(['Opening ' d(i).name '...']);
        load(d(i).name);
        answers = [answers, solutions];
    end
    cd('..');
end
\end{minted}
\normalsize

\section{reFitter.m}
\small
\begin{minted}{matlab}
function fixed=reFitter(broked, r, params)
    fixed = broked;
    for i=1:length(fixed),
        fixed{i} = cellfun(@(kset) { ...
            fitHelper(kset{2}, r, params), ...
            kset{2}, kset{3}}, fixed{i}, 'UniformOutput', false);
    end
end

function fitted=fitHelper(tT, r, params)
    fitted = fitter(tT(1,:),tT(2,:),r, params);
end
\end{minted}
\normalsize

\section{analyzer.m}
\small
\begin{minted}{matlab}
%analyzer
%Does some analyzing of the simulation results
%breaks for [kxy,kz] != meshgrid(ks,ks)

%Solutions location
%load solutions-19-Sep-2010/solutions-all.mat;

%Things I already know :)
%ks = linspace(0.2, 0.4, 6);
%ks = [0.2,0.4];
%[kxy, kz] = meshgrid(ks, ks);
%[kzy, kz] = meshgrid(0.3, 0.5);
%ks = 1;
%angles = 0:15:90;
%angles = 0:5:90;
%angles = [0 90];

%For an obvious color gradient, from red to blue right now.
colores = @(i,n) [sin((i/n)*pi/2), 0, cos((i/n)*pi/2)];

disp(['Showing overlaid plots (YES ALL OF THEM)' ...
      ' to make sure they "look" right:']);
figure;
hold on;
for theta = 1:length(angles)
    for i=1:length(ks)^2


        tT = answers{theta}{i}{2};
        plot(log(tT(1,tT(1,:) > 1 )),tT(2,tT(1,:) > 1), ...
             'color', colores(i,length(ks)^2));
    end
end

disp('Sanity checking results for isotropic cases');
figure;
hold on;
kmsold = 0 * cellfun(@(prison) prison{1}, answers{1});
for i=1:length(angles)
    %Extracts all the measured k's from the data
    % "prison" refers to cell representing particular
    % k combination in answers{theta}
    kms = cellfun(@(prison) prison{1}, answers{i});
    if kms == kmsold,
        disp('wtf exactly equivalent kms''s');
    end
    %diag(kms)
    %diag(kxy)
    errs = 100*(diag(kms) - diag(kxy))./diag(kxy);
    %Not necessary to be 3d anymore :)
    plot3(diag(kxy), errs, angles(i)*ones(size(diag(kxy))), ...
          '*-', 'color', colores(i,length(angles)) );
    xlabel('k_{actual}');
    ylabel('error (%)');
    zlabel('angle (degrees)');
end

disp('Figuring out T_surf_avg at time T:');
%figure;
%hold on;
for theta=1:length(angles)
    tavgs = cellfun(@(prison) prison{3}, answers{theta});
    try
        assert(all(all(tavgs< 0.001)));
    catch
        disp(['Warning: average surface temps are a bit high' ...
              ' at theta=' num2str(angles(theta))] );
        disp(tavgs);
    end
    if theta == length(angles)
        figure;
        hold on;
        contourf(kxy,kz,tavgs);
        colorbar;
        colormap('pink');
        title(['Average Surface Temperature at End of '...
               'Heating Curve Simulation for a representative angle');
        xlabel('K_{xy}');
        ylabel('K_{zz}');
    end
end

%dimensions changed to be in order kxy, then
%rows are angle and columns are kzz
disp('Plotting k_{meas}/k_{xy} vs. \theta and k_{z}/k_{xy}...');
kms=cell(size(ks));
for i=1:length(angles)
    kmsbyangle = cellfun(@(prison) prison{1}, answers{i});
    for j=1:length(ks)
        %Normalize by particular kxy
        kms{j} = [kms{j}; kmsbyangle(:,j)'/ks(j)];
    end
end
figure;
hold on;
[kgrid, anggrid] = meshgrid(ks, angles);
for n=1:length(ks)
    x = reshape(anggrid',[],1), reshape(kgrid'/ks(n),[],1)
    y = reshape(kms{n}',[],1)
    plot3(x, y, '*-', 'color', colores(n, length(ks)));
end
%legend(arrayfun(@(x) num2str(x),ks, 'UniformOutput', false));
xlabel('angle (degrees)');
ylabel('k_{zz}/k_{xy}');
zlabel('k_{meas}/k_{xy}');
\end{minted}
\normalsize

\section{tabulator.m}
\small
\begin{minted}{matlab}
%tabulator
%turns lame structures into some csv action

%given params:
%  answers
%  angles
%  kxy
%  kz

% Bad style, but I'm dealing with a cluttered namespace
% because I'm not functionalizing these.
% This is because parameter passing is annoying. So, leave me alone.
[Kxy, Kz] = meshgrid(kxy, kz);

fprintf('angle, kxy, kz, kmeas \n');
for t=1:length(angles),
    for pt=1:size(Kxy,1)*size(Kxy,2),
        fprintf([ num2str(angles(t)) ', ' ...
                  num2str(Kxy(pt)) ', ' ...
                  num2str(Kz(pt)) ', ' ...
                  num2str(answers{t}{pt}{1}) '\n']);
    end
end
\end{minted}
\normalsize

\section{symmetric\_tocell.m}
\small
\begin{minted}{matlab}
function a=symmetric_tocell(A)
    % Converts a symmetric matrix A into a cell 
    % containing the values, comsol-style.
    % I need to make sure comsol likes them nx1 instead of 1xn.

    %test for squareness
    assert(size(A,1)==size(A,2), ...
           'Dawg that matrix aint square much less symmetric');
    %test for symmetry
    for i=1:size(A,1),
        for j=i:size(A,1),
            if (A(i,j)~=A(j,i)),
                disp(['Warning: Dawg that matrix aint square (A(' ...
                      int2str(i) int2str(j) ')=' num2str(A(i,j)) ...
                      ' != A(' int2str(j) int2str(i) ...
                      ')=' num2str(A(j,i)) ' ! )']);
            end
        end
    end

    %The actual heavy lifting.
    a={};
    for m=1:size(A,1),
        %Takes upper-triangular section of mth column
        for element=A(1:m,m)'
            a{size(a,1)+1,1}=element;
        end
    end

end

function t=triangle(n)
    t=(n^2+n)/2;
end
\end{minted}
\normalsize

